\section{Introduction}

The practice of evidence-based medicine requires the integration of clinical expertise, patient values, and the best available research evidence. In lifestyle medicine, where interventions focus on modifiable behaviors such as nutrition, physical activity, and stress management, maintaining current knowledge of the evidence base presents significant challenges \cite{clayton2023foundations}. The rapid evolution of research findings, combined with the complexity of lifestyle interventions and individual patient variations, creates substantial uncertainty in clinical decision-making \cite{santos2024improving}.

This paper presents a methodological framework for addressing this uncertainty through a Living Systematic Review (LSR) approach combined with Bayesian statistical methods and implementation through standardized clinical decision support mechanisms \cite{santos2024healthcare}. Our framework creates a continuous pathway from evidence uncertainty to practical clinical decision support, leveraging both methodological advances and modern health informatics standards.

Living Systematic Reviews represent an evolution of traditional systematic reviews, featuring continual updating as new evidence emerges \cite{elliott2014living}. This approach is particularly valuable in fields with rapidly evolving evidence bases, such as lifestyle medicine. By maintaining an up-to-date synthesis of research findings, LSRs provide clinicians with access to current evidence, reducing the common lag between research publication and clinical implementation \cite{elliott2017living}.

Bayesian statistical methods offer a natural framework for managing uncertainty in clinical evidence \cite{spiegelhalter2004bayesian}. Unlike traditional frequentist approaches, Bayesian methods explicitly incorporate prior knowledge and update beliefs as new evidence emerges. This aligns conceptually with clinical reasoning, where clinicians integrate their prior knowledge with new information about individual patients. In our framework, Bayesian methods provide the mathematical foundation for evidence synthesis and uncertainty quantification.

For modeling disease progression and the impact of interventions over time, Markov models offer a powerful approach \cite{sonnenberg1993markov}. These models represent health states and transitions between them, allowing for the simulation of disease trajectories under different intervention scenarios. By incorporating Markov models into our framework, we enable the projection of long-term outcomes based on current evidence, supporting both clinical decision-making and patient education about potential benefits of lifestyle modifications.

The translation of evidence into practice requires not only methodological rigor but also effective implementation tools. Our framework leverages two complementary standards for clinical decision support implementation: openEHR Decision Language \cite{beale2021decision} and HL7 FHIR-based decision support mechanisms \cite{lichtner2023representation}. The openEHR Decision Language provides a structured approach to representing clinical rules and guidelines, while FHIR-based mechanisms enable integration with diverse health information systems.

Decision Logic Modules (DLMs) in the openEHR ecosystem enable the representation of clinical knowledge in a form that can be directly executed within health information systems \cite{silva2024openehr}. These modules support sophisticated decision rules while maintaining readability for clinical domain experts. By implementing evidence-based guidelines as DLMs, our framework bridges the gap between evidence synthesis and clinical application.

The cellular mechanisms of oxygen sensing, recognized by the 2019 Nobel Prize in Physiology or Medicine, illustrate the fundamental importance of physiological monitoring in health management \cite{nobel2019physiology}. The Hypoxia-Inducible Factor (HIF) pathway demonstrates how cellular responses to oxygen levels influence metabolism and disease development \cite{bosco2020mitochondrial}. Our framework incorporates this understanding by connecting physiological measurements, such as oxygen saturation, to evidence-based interventions through formal decision support systems.

In this paper, we describe the components of our methodological framework, including the structure of our Living Systematic Review approach, the application of Bayesian methods and Markov models, and the implementation of clinical decision support using standardized languages. We also present a case example in the domain of physical activity interventions, demonstrating how our framework can be applied to synthesize evidence and translate it into executable decision support logic. Through this integrated approach to evidence synthesis and decision support, we aim to improve the application of lifestyle medicine principles in clinical practice \cite{santos2023improving}.
